\section{UDP}
\subsection{The Assignment}
    \subsubsection*{Problems}
    \begin{enumerate}[label=\bfseries Problem \arabic*:,leftmargin=*,labelindent=1em]
    %%%%%%%%%%%%%%%%%%%%%%%%%%%%%%%%%%%%%%%%%%%%%%%%% Problem 2-1
        \item Select one UDP packet from you r trace . From this packet, determine how many fields there are in the UDP header.\\[0.2mm]
        \soln There are 4 fields. : Source Port, Destination Port, Length, and Checksum
    %     \vspace{-4mm}  
        \begin{figure}[!h]\centering
        \hspace{15mm}  
    		\includegraphics[width=.85\textwidth]{image/week02/2-1-1.png}
    		\caption{\footnotesize Problem 2-1's screenshot : }
    		\vspace{-10pt}
        \end{figure}
    %%%%%%%%%%%%%%%%%%%%%%%%%%%%%%%%%%%%%%%%%%%%%%%%% Problem 2-2
        \item By consulting the displayed information in Wireshark’s this packet packet content field for ,determine the length (in bytes) of each of the UDP header fields.\\[0.2mm]
        \soln The header length of UDP is that $\text{Length} - \text{UDP payload} : 58 - 50 = 8$.\\
        As we can see in figure the 4 fields of header has the same length. Therefore each of 4 header fields is 2 bytes long.
        \begin{figure}[h!]
        \centering
        \subfloat[ ]{
            \includegraphics[width=0.48\textwidth] {image/week02/2-2-1.png}
        }\hspace{2mm}
        \subfloat[]{
            \includegraphics[width=0.48\textwidth] {image/week02/2-2-2.png}
        }
        \hfill
        \subfloat[ ]{
            \includegraphics[width=0.48\textwidth] {image/week02/2-2-3.png}
        }\hspace{2mm}
        \subfloat[]{
            \includegraphics[width=0.48\textwidth] {image/week02/2-2-4.png}
        }
		\caption{\footnotesize Problem 2-2's screenshot : }
        \end{figure}
    %     \vspace{-4mm}  
    %     \begin{figure}[!h]\centering
    %     \hspace{15mm}  
    % 		\includegraphics[width=.85\textwidth]{image/week02/ }
    % 		\caption{\footnotesize Problem 2-2's screenshot : }
    % 		\vspace{-10pt}
    %     \end{figure}
    %%%%%%%%%%%%%%%%%%%%%%%%%%%%%%%%%%%%%%%%%%%%%%%%% Problem 2-3
\clearpage
        \item The value in the Length field is the length of what? this answer\\[0.2mm]
        \soln Length’ is the length of the UDP header plus the UDP data. We can verify this from the packet below.  
        $\text{Total length} = \text{header} + \text{data} = 8 + 50 = 58 \text{(bytes)}$
    %     \vspace{-4mm}  
        \begin{figure}[!h]\centering
        \hspace{15mm}  
    		\includegraphics[width=.85\textwidth]{image/week02/2-3-1.png}
    		\caption{\footnotesize Problem 2-3's screenshot : }
    		\vspace{-10pt}
        \end{figure}
% \newpage
    %%%%%%%%%%%%%%%%%%%%%%%%%%%%%%%%%%%%%%%%%%%%%%%%% Problem 2-4
        \item What is the maximum number of bytes that c\\[0.2mm]
        \soln UDP header’s length field is 2 bytes (16 bit) long, so UDP’s maximum length is $2^{16} - 1 = 65535$ bytes. Since UDP header is 8 bytes long, UDP payload’s maximum length is $65535 - 8 = 65537$ bytes .
    %     \vspace{-4mm}  
    %     \begin{figure}[!h]\centering
    %     \hspace{15mm}  
    % 		\includegraphics[width=.85\textwidth]{image/week02/ }
    % 		\caption{\footnotesize Problem 2-4's screenshot : }
    % 		\vspace{-10pt}
    %     \end{figure}
    %%%%%%%%%%%%%%%%%%%%%%%%%%%%%%%%%%%%%%%%%%%%%%%%% Problem 2-5
        \item What is the largest possible source port number?\\[0.2mm]
        \soln UDP header’s source port field is 2 bytes (16 bit) long, so the largest possible source port number is $2^{16} - 1 = 65535$.
    % \vspace{-4mm}  
    %     \begin{figure}[!h]\centering
    %     \hspace{15mm}  
    % 		\includegraphics[width=.85\textwidth]{image/week02/ }
    % 		\caption{\footnotesize Problem 2-5's screenshot : }
    % 		\vspace{-10pt}
    %     \end{figure}
    %%%%%%%%%%%%%%%%%%%%%%%%%%%%%%%%%%%%%%%%%%%%%%%%% Problem 2-6
        \item What is the protocol number for UDP? 
        Give your answer in both hex decimal notation. 
        To answer this question, you’ll need to loo adecimal and k into the field of the IP datagram containing this UDP segment.\\[0.2mm]
        \soln hexadecimal notaion : 11  /  decimal notation : 17
    %     \vspace{-4mm}  
        \begin{figure}[!h]\centering
        \hspace{15mm}  
    		\includegraphics[width=.85\textwidth]{image/week02/2-6-1}
    		\caption{\footnotesize Problem 2-6's screenshot : }
    		\vspace{-10pt}
        \end{figure}
    \end{enumerate}
\newpage