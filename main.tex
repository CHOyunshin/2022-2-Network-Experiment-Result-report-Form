%페이지 세팅
% 한국어 Setting의 경우 10 pt 이용해주자. 또한 줄간격인 line spread 도 1.15로 설정 
\documentclass[9.5pt,a4paper]{article}
\setlength{\parindent}{0em}                  %DISTANCIA SANGRÍA
\setlength{\parskip}{0.5em}                  %DISTANCIA ENTRE PÁRRAFOS
\linespread{1.2}
\textwidth 6.5in
\textheight 9.in
\oddsidemargin 0in
\headheight 0in
\usepackage{amsmath}
\usepackage{tcolorbox}
\usepackage{amssymb}
\usepackage{amsthm}
\usepackage{lastpage}
\usepackage{fancyhdr}
\usepackage{accents}
\usepackage{setup}
\usepackage{import}
\usepackage{fancyhdr}
\usepackage{layouts}
\addtolength{\voffset}{0mm}
\addtolength{\textheight}{0mm}

\usepackage{xcolor}
\usepackage{mdframed}
\usepackage[shortlabels]{enumitem}
\usepackage{indentfirst}
\usepackage{hyperref}


\usepackage{enumitem}
\newlist{enumerateoptional}{enumerate}{1}
\setlist[enumerateoptional]{
    before=\let\item\optionalItem,
    label=\arabic*,
    nosep,
    labelindent=20mm,
    leftmargin=*
}
\makeatletter
%%%%%%%%%%%%%%%%%%%%%%%%%%%%%%%%%%%%%%%%%%%%%%%%%%%%%%%%%%%%%%%%%%%%%%%%%%
% Sectioning Sytle Change
\renewcommand{\thesubsection}{\thesection. \arabic{subsection}}
\renewcommand{\thesubsubsection}{\thesection. \arabic{subsection}. \arabic{subsubsection}}
% \renewcommand{\thesubsection}{\thesection.\alph{subsection}}
% Customize Section Style
% we use \prefix@<level> only if it is defined
\renewcommand{\@seccntformat}[1]{%
  \ifcsname prefix@#1\endcsname
    \csname prefix@#1\endcsname
  \else
    \csname the#1\endcsname\quad
  \fi}
% define \prefix@section
% 11 / 14 실험 제목 생략을 위해서 잠시 수정해줌 
\newcommand\prefix@section{Experiment  \thesection}
% \newcommand\prefix@subsection{ \thesubsection\ \ }
% \newcommand\prefix@subsubsection{\quad \thesubsubsection\ \ }
\makeatother
%%%%%%%%%%%%%%%%%%%%%%%%%%%%%%%%%%%%%%%%%%%%%%%%%%%%%%%%%%%%%%%%%%%%%%%%%%
\let\realItem\item
% Setting Table Color
% \usepackage[table,xcdraw]{xcolor}
\newcommand\optionalItem[1][]{%
  \refstepcounter{enumerateoptionali}% increment the counter
  \realItem[\bfseries#1~\theenumerateoptionali)]%
}
% WEEK 06 파이썬 코드 의 listing 을 위해서 수정해주는 이름 부분 
% \usepackage{listings}
% \renewcommand{\lstlistingname}{mininet python API custom code}
% \renewcommand{\figcaption}{\footnotesize \textbf{mininet API} python custom code}
\renewcommand{\listingscaption}{\footnotesize \textbf{OMNet++ Source Code}}

%%%%%%%%%%%%%%%%%%%%%%%%%%%%%%%%%%%%%%%%%%%%%%%%%%%%%%%%%%%%%%%%%%%%%%%%%%
    % 매주 리포트를 작성할때 이 부분을 수정하면 보고서 전체가 수정된다. 
%%%%%%%%%%%%%%%%%%%%%%%%%%%%%%%%%%%%%%%%%%%%%%%%%%%%%%%%%%%%%%%%%%%%%%%%%%
% 작성하는 주차
\newcommand{\numnum}{11}
% 실험 부제목
% \newcommand{\subtitle}{wireshark(1) : Getting started HTTP DNS}
% \newcommand{\subtitle}{wireshark(2) : TCP UDP IP protocols}
% \newcommand{\subtitle}{USPR(1) : Introduction, Labview, Spectogram}
% \newcommand{\subtitle}{USPR(2) : Active Sensing}
% \newcommand{\subtitle}{USPR(3) : CNN Classifier}
% \newcommand{\subtitle}{mininet (2) : sdn-based Routing}
% \newcommand{\subtitle}{OMNet++ Simulator Basic}
\newcommand{\subtitle}{Mobile Communication and Cellular Network}
%-------------------------------------------------------------------------
%--> 떠다니는 객체 사용하는 부분 
\usepackage{wrapfig}
% --> Problem numbering 부분의 수정 필요 
% \newenvironment{problem}[2][Problem]  
%     { \begin{mdframed}[backgroundcolor=gray!20] \textbf{#1 #2} \\}
%     {  \end{mdframed}}
% Enumerate Listing modified 
% \renewcommand{\labelenumi}{Problem  \theenumi}
\newcommand{\soln}[1][Answer]{\noindent\textbf{#1}\quad}
% \newenvironment{problem}[2][Problem]  
%     { \begin{mdframed}[backgroundcolor=gray!20] \textbf{#1 #2} \\}
%     {  \end{mdframed}}
%-------------------------------------------------------------------------
\begin{document}
\pagestyle{fancy}
\fancyhf{}
    \rhead{\small 2조 2016142096 조윤신, 2017142043 김재민}
    \lhead{\textsc{Result Report Week\ \numnum\ \subtitle}}
\cfoot{\thepage}
\renewcommand\headrulewidth{0.3mm}
%\renewcommand\footrulewidth{0.3mm}
\thispagestyle{plain}
\begin{flushleft}
\textsc{School of Electrical and Electronic of Enginnering} \\
\textsc{eee 4474-01 : Experiments on Communication Networks }\\[0.1cm]
\small{\textsc{\textbf{2조} 2016142096 \textbf{조윤신}, 2017142043 \textbf{김재민}}}\\
\end{flushleft}

\begin{flushright}\vspace{-25mm}
    \includegraphics[height=3cm]{image/1.jpg}
    \vspace{5mm}
\end{flushright}
    \begin{center}\vspace{-1.5cm}
        \textbf{\huge  Result Report Week \numnum}\\ 
        \vspace{1.8mm}
        {\large \subtitle}\\           
    \end{center}
\vspace{-5mm}
\rule{\linewidth}{0.4mm}
\vspace{-5mm}
%-------------------------------------------------------------------------
% main documents
% 모든 프로그래밍 편집기 (IDE)에서 원도우 기준으로 '컨트롤 + /'를 누르면 
% 커서가 활성화된 line의 주석처리를 해제 지정할 수 있다. 
%%%%%%%%%%%%%%%%%%%%%%%%%%%%%%%%%%%%%%%%%%%%%%%%%%%%%%%%%%%%%%%%%%%%%%%%%%
                        % For Week 01 ~ 09.14
%%%%%%%%%%%%%%%%%%%%%%%%%%%%%%%%%%%%%%%%%%%%%%%%%%%%%%%%%%%%%%%%%%%%%%%%%%
% \import{./TEX/week01}{01_exp1}
% \import{./TEX/week01}{02_exp2}
% \import{./TEX/week01}{03_exp3}
%-------------------------------------------------------------------------
%%%%%%%%%%%%%%%%%%%%%%%%%%%%%%%%%%%%%%%%%%%%%%%%%%%%%%%%%%%%%%%%%%%%%%%%%%
                        % For Week 02 ~ 09.21
%%%%%%%%%%%%%%%%%%%%%%%%%%%%%%%%%%%%%%%%%%%%%%%%%%%%%%%%%%%%%%%%%%%%%%%%%%
% \import{./TEX/week02}{01_01_tcp}
% \import{./TEX/week02}{01_02_tcp_basic}
% \import{./TEX/week02}{01_03_tcp_congestion}
% \import{./TEX/week02}{02_udp}
% \import{./TEX/week02}{03_01_ip}
% \import{./TEX/week02}{03_02_basic_ipv4}
% \import{./TEX/week02}{03_03_fragmentation}
%-------------------------------------------------------------------------
%%%%%%%%%%%%%%%%%%%%%%%%%%%%%%%%%%%%%%%%%%%%%%%%%%%%%%%%%%%%%%%%%%%%%%%%%%
                        % For Week 03 ~ 09.28
%%%%%%%%%%%%%%%%%%%%%%%%%%%%%%%%%%%%%%%%%%%%%%%%%%%%%%%%%%%%%%%%%%%%%%%%%%
% \import{./TEX/week03}{01-visualization}
% \import{./TEX/week03}{02-tx}
% \import{./TEX/week03}{03-other-txs}
%-------------------------------------------------------------------------
%%%%%%%%%%%%%%%%%%%%%%%%%%%%%%%%%%%%%%%%%%%%%%%%%%%%%%%%%%%%%%%%%%%%%%%%%%
                        % For Week 04 ~ 10.05
%%%%%%%%%%%%%%%%%%%%%%%%%%%%%%%%%%%%%%%%%%%%%%%%%%%%%%%%%%%%%%%%%%%%%%%%%%
% \import{./TEX/week04}{00_notice}
% \import{./TEX/week04}{01-spectrogram}
% \newpage
% \import{./TEX/week04}{02_}
% \newpage
% \import{./TEX/week04}{03_}
%-------------------------------------------------------------------------
%%%%%%%%%%%%%%%%%%%%%%%%%%%%%%%%%%%%%%%%%%%%%%%%%%%%%%%%%%%%%%%%%%%%%%%%%%
                        % For Week 05 ~ 10.12
%%%%%%%%%%%%%%%%%%%%%%%%%%%%%%%%%%%%%%%%%%%%%%%%%%%%%%%%%%%%%%%%%%%%%%%%%%
% \import{./TEX/week05}{00_notice}
% \import{./TEX/week05}{01}
% \import{./TEX/week05}{02}
% \import{./TEX/week05}{03}
% \import{./TEX/week05}{04_01}
% \import{./TEX/week05}{04_02}
%-------------------------------------------------------------------------
%%%%%%%%%%%%%%%%%%%%%%%%%%%%%%%%%%%%%%%%%%%%%%%%%%%%%%%%%%%%%%%%%%%%%%%%%%
                        % For Week 06 ~ 10.26 (중간고사 일정 변경)
%%%%%%%%%%%%%%%%%%%%%%%%%%%%%%%%%%%%%%%%%%%%%%%%%%%%%%%%%%%%%%%%%%%%%%%%%%
% \import{./TEX/week06}{00}
% \import{./TEX/week06}{01}
% \import{./TEX/week06}{02}
% \import{./TEX/week06}{03}
%-------------------------------------------------------------------------
%%%%%%%%%%%%%%%%%%%%%%%%%%%%%%%%%%%%%%%%%%%%%%%%%%%%%%%%%%%%%%%%%%%%%%%%%%
                        % For Week 08 ~ 11.02
%%%%%%%%%%%%%%%%%%%%%%%%%%%%%%%%%%%%%%%%%%%%%%%%%%%%%%%%%%%%%%%%%%%%%%%%%%
% \import{./TEX/week08}{01}
% \import{./TEX/week08}{02}
% \import{./TEX/week08}{03}
% \import{./TEX/week08}{04}
% \import{./TEX/week08}{05}
%-------------------------------------------------------------------------
%%%%%%%%%%%%%%%%%%%%%%%%%%%%%%%%%%%%%%%%%%%%%%%%%%%%%%%%%%%%%%%%%%%%%%%%%%
                        % For Week 09 ~ 11.09
%%%%%%%%%%%%%%%%%%%%%%%%%%%%%%%%%%%%%%%%%%%%%%%%%%%%%%%%%%%%%%%%%%%%%%%%%%
% \import{./TEX/week09}{01}
% \import{./TEX/week09}{02}
% \import{./TEX/week09}{03}
%%%%%%%%%%%%%%%%%%%%%%%%%%%%%%%%%%%%%%%%%%%%%%%%%%%%%%%%%%%%%%%%%%%%%%%%%%
                        % For Week 10 ~ 11.16
%%%%%%%%%%%%%%%%%%%%%%%%%%%%%%%%%%%%%%%%%%%%%%%%%%%%%%%%%%%%%%%%%%%%%%%%%%
% \import{./TEX/week10}{01}
% \import{./TEX/week10}{02}
% \import{./TEX/week10}{03}
%-------------------------------------------------------------------------
%%%%%%%%%%%%%%%%%%%%%%%%%%%%%%%%%%%%%%%%%%%%%%%%%%%%%%%%%%%%%%%%%%%%%%%%%%
                        % For Week 11 ~ 11.23 -> 조윤신 
% notion Link : https://www.notion.so/Week-11-7773a1116c864b7baaafebd13b574259
%%%%%%%%%%%%%%%%%%%%%%%%%%%%%%%%%%%%%%%%%%%%%%%%%%%%%%%%%%%%%%%%%%%%%%%%%%
\import{./TEX/week11}{01}
\import{./TEX/week11}{02}
% \import{./TEX/week11}{03}
%-------------------------------------------------------------------------
%%%%%%%%%%%%%%%%%%%%%%%%%%%%%%%%%%%%%%%%%%%%%%%%%%%%%%%%%%%%%%%%%%%%%%%%%%
                        % For Week 12 ~ 11.30 -> 김재민 
                
%%%%%%%%%%%%%%%%%%%%%%%%%%%%%%%%%%%%%%%%%%%%%%%%%%%%%%%%%%%%%%%%%%%%%%%%%%
% \import{./TEX/week12}{01}
% \import{./TEX/week12}{02}
% \import{./TEX/week12}{03}
%-------------------------------------------------------------------------
\end{document}