%페이지 세팅
\documentclass[10pt,a4paper]{article}
\setlength{\parindent}{0em}                  %DISTANCIA SANGRÍA
\setlength{\parskip}{0.5em}                  %DISTANCIA ENTRE PÁRRAFOS
\textwidth 6.5in
\textheight 9.in
\oddsidemargin 0in
\headheight 0in

\usepackage{amsmath}
\usepackage{tcolorbox}
\usepackage{amssymb}
\usepackage{amsthm}
\usepackage{lastpage}
\usepackage{fancyhdr}
\usepackage{accents}
\usepackage{setup}
\usepackage{import}
\usepackage{fancyhdr}
\usepackage{layouts}
\addtolength{\voffset}{0mm}
\addtolength{\textheight}{0mm}

\usepackage{xcolor}
\usepackage{mdframed}
\usepackage[shortlabels]{enumitem}
\usepackage{indentfirst}
\usepackage{hyperref}

\usepackage{enumitem}
\newlist{enumerateoptional}{enumerate}{1}
\setlist[enumerateoptional]{
    before=\let\item\optionalItem,
    label=\arabic*,
    nosep,
    labelindent=20mm,
    leftmargin=*
}
\let\realItem\item

\newcommand\optionalItem[1][]{%
  \refstepcounter{enumerateoptionali}% increment the counter
  \realItem[\bfseries#1~\theenumerateoptionali)]%
}
\renewcommand{\thesubsection}{\thesection.\alph{subsection}}
%%%%%%%%%%%%%%%%%%%%%%%%%%%%%%%%%%%%%%%%%%%%%%%%%%%%%%%%%%%%%%%%%%%%%%%%%%
    % 매주 리포트를 작성할때 이 부분을 수정하면 보고서 전체가 수정된다. 
%%%%%%%%%%%%%%%%%%%%%%%%%%%%%%%%%%%%%%%%%%%%%%%%%%%%%%%%%%%%%%%%%%%%%%%%%%
% 작성하는 주차
\newcommand{\numnum}{2}
% 실험 부제목
% \newcommand{\subtitle}{wireshark(1) : Getting started HTTP DNS}
\newcommand{\subtitle}{wireshark(2) : TCP UDP IP protocols}
%-------------------------------------------------------------------------
%--> 떠다니는 객체 사용하는 부분 
\usepackage{wrapfig}
\newenvironment{problem}[2][Problem]  
    { \begin{mdframed}[backgroundcolor=gray!20] \textbf{#1 #2} \\}
    {  \end{mdframed}}
% Enumerate Listing modified 
\renewcommand{\labelenumi}{Problem  \theenumi}
\newcommand{\soln}[1][Answer]{\noindent\textbf{#1}  }
% \newenvironment{problem}[2][Problem]  
%     { \begin{mdframed}[backgroundcolor=gray!20] \textbf{#1 #2} \\}
%     {  \end{mdframed}}
%-------------------------------------------------------------------------
\begin{document}
\pagestyle{fancy}
\fancyhf{}
    \rhead{\small 2조 2016142096 조윤신, 2017142043 김재민}
    \lhead{\textsc{Result Report Week\numnum \subtitle}}
\cfoot{\thepage}
\renewcommand\headrulewidth{0.3mm}
%\renewcommand\footrulewidth{0.3mm}
\thispagestyle{plain}
\begin{flushleft}
\textsc{School of Electrical and Electronic of Enginnering} \\
\textsc{eee 4474-01 : Experiments on Communication Networks }\\[0.1cm]
\small{\textsc{\textbf{2조} 2016142096 \textbf{조윤신}, 2017142043 \textbf{김재민}}}\\
\end{flushleft}

\begin{flushright}\vspace{-25mm}
    \includegraphics[height=3cm]{image/1.jpg}
    \vspace{5mm}
\end{flushright}
    \begin{center}\vspace{-1.5cm}
        \textbf{\huge  Result Report Week \numnum}\\ 
        \vspace{0.6mm}
        {\large \subtitle}\\           
    \end{center}
\vspace{-5mm}
\rule{\linewidth}{0.4mm}
\vspace{-5mm}
%-------------------------------------------------------------------------
% main documents
% 모든 프로그래밍 편집기 (IDE)에서 원도우 기준으로 '컨트롤 + /'를 누르면 
% 커서가 활성화된 line의 주석처리를 해제 지정할 수 있다. 
%%%%%%%%%%%%%%%%%%%%%%%%%%%%%%%%%%%%%%%%%%%%%%%%%%%%%%%%%%%%%%%%%%%%%%%%%%
                        % For Week 01 ~ 09.14
%%%%%%%%%%%%%%%%%%%%%%%%%%%%%%%%%%%%%%%%%%%%%%%%%%%%%%%%%%%%%%%%%%%%%%%%%%
% \import{./TEX/week01}{01_exp1}
% \import{./TEX/week01}{02_exp2}
% \import{./TEX/week01}{03_exp3}
%-------------------------------------------------------------------------
%%%%%%%%%%%%%%%%%%%%%%%%%%%%%%%%%%%%%%%%%%%%%%%%%%%%%%%%%%%%%%%%%%%%%%%%%%
                        % For Week 02 ~ 09.21
%%%%%%%%%%%%%%%%%%%%%%%%%%%%%%%%%%%%%%%%%%%%%%%%%%%%%%%%%%%%%%%%%%%%%%%%%%
\import{./TEX/week02}{01_01_tcp}
\import{./TEX/week02}{01_02_tcp_basic}
\import{./TEX/week02}{01_03_tcp_congestion}
\import{./TEX/week02}{02_udp}
\import{./TEX/week02}{03_01_ip}
\import{./TEX/week02}{03_02_basic_ipv4}
\import{./TEX/week02}{03_03_fragmentation}
%-------------------------------------------------------------------------
%%%%%%%%%%%%%%%%%%%%%%%%%%%%%%%%%%%%%%%%%%%%%%%%%%%%%%%%%%%%%%%%%%%%%%%%%%
                        % For Week 03 ~ 09.28
%%%%%%%%%%%%%%%%%%%%%%%%%%%%%%%%%%%%%%%%%%%%%%%%%%%%%%%%%%%%%%%%%%%%%%%%%%
% \import{./TEX/week03}{01_}
% \newpage
% \import{./TEX/week03}{02_}
% \newpage
% \import{./TEX/week03}{03_}
%-------------------------------------------------------------------------
%%%%%%%%%%%%%%%%%%%%%%%%%%%%%%%%%%%%%%%%%%%%%%%%%%%%%%%%%%%%%%%%%%%%%%%%%%
                        % For Week 04 ~ 10.05
%%%%%%%%%%%%%%%%%%%%%%%%%%%%%%%%%%%%%%%%%%%%%%%%%%%%%%%%%%%%%%%%%%%%%%%%%%
% \import{./TEX/week04}{01_}
% \newpage
% \import{./TEX/week04}{02_}
% \newpage
% \import{./TEX/week04}{03_}
%-------------------------------------------------------------------------
%%%%%%%%%%%%%%%%%%%%%%%%%%%%%%%%%%%%%%%%%%%%%%%%%%%%%%%%%%%%%%%%%%%%%%%%%%
                        % For Week 05 ~ 10.12
%%%%%%%%%%%%%%%%%%%%%%%%%%%%%%%%%%%%%%%%%%%%%%%%%%%%%%%%%%%%%%%%%%%%%%%%%%
% \import{./TEX/week05}{01_}
% \newpage
% \import{./TEX/week05}{02_}
% \newpage
% \import{./TEX/week05}{03_}
%-------------------------------------------------------------------------
%%%%%%%%%%%%%%%%%%%%%%%%%%%%%%%%%%%%%%%%%%%%%%%%%%%%%%%%%%%%%%%%%%%%%%%%%%
                        % For Week 06 ~ 10.19
%%%%%%%%%%%%%%%%%%%%%%%%%%%%%%%%%%%%%%%%%%%%%%%%%%%%%%%%%%%%%%%%%%%%%%%%%%
% \import{./TEX/week06}{01_}
% \newpage
% \import{./TEX/week06}{02_}
% \newpage
% \import{./TEX/week06}{03_}
%-------------------------------------------------------------------------
%%%%%%%%%%%%%%%%%%%%%%%%%%%%%%%%%%%%%%%%%%%%%%%%%%%%%%%%%%%%%%%%%%%%%%%%%%
                        % For Week 07 ~ 10.26
%%%%%%%%%%%%%%%%%%%%%%%%%%%%%%%%%%%%%%%%%%%%%%%%%%%%%%%%%%%%%%%%%%%%%%%%%%
% \import{./TEX/week07}{01_}
% \newpage
% \import{./TEX/week07}{02_}
% \newpage
% \import{./TEX/week07}{03_}
%-------------------------------------------------------------------------
\end{document}